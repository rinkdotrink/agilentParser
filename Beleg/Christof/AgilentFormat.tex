\section{Das Agilent-Format}\label{AgilentFormat}
\subsection{Struktur und Syntax}\label{strukturundsyntax}
Testdaten im Agilent-Format  werden in einer Datei als eine Folge von sog. \textit{log-records} gespeichert. Jedes log-record ist durch geschweifte Klammern umschlossen und beginnt mit einen Pr�fix, welches aus einem @-Zeichen besteht, gefolgt von beschreibenden Zeichen, die den Typ des Records eindeutig identifizieren (z.B. @A-JUM). Daraufhin folgt eine bestimmte Anzahl von Datenfeldern, die alphanumerische Messwerte im ASCII-Format enthalten und voneinander durch das \textbar-Zeichen(Pipe) getrennt sind. Sollten einem Datenfeld keine Messwerte zugeordnet worden sein, so wird dies einfach durch zwei nebeneinander stehende Pipes dargestellt(z.B. \textbar \textbar). Falls davon das letzte Datenfeld eines Records betroffen sein sollte, so steht hinter dem letzen Pipe kein weiteres Zeichen(z.B. 501-6338-02\textbar RevA\textbar \textbar \textbar \textbar ) und das Ende des Records wird mit der schlie�enden Klammer ''\}'' markiert oder aber ein neues Subrecord mit der �ffnenden Klammer ''\{'' begonnen.   
Die log-records sind hierarchisch angeordnet in \textit{records} und \textit{subrecords}. Die subrecords dienen dazu  das vorhergehende record genauer zu beschreiben.  Dabei ist jedem subrecord h�chstens ein record unmittelbar �bergeordnet und somit kann man die Log-Datei als einen Baum darstellen um zwischen den einzelnen Knoten zu navigieren und sie auszulesen. \\
Das Wurzelelement jeder der uns zur Verf�gung stehenden Log-Datei ist das \\ @BATCH-Record, welches erst vollst�ndig durch das @BTEST-Record beschrieben wird und dieses als subrecord mit geschweiften Klammern umschlie�t. Jede Log-Datei enth�lt jeweils ein @BATCH-Record und ein @BTEST-Record.  Dies ist die �bliche Struktur  und sie wurde in allen der uns zur Verf�gung stehenden Log-Dateien befolgt. Je nach Testsystem k�nnte es jedoch evtl. zu Abweichungen von dieser Struktur kommen. Es k�nnten beispielsweise mehrere @BATCH-Records in einer Log-Datei gespeichert werden.   
Das @BTEST-Record kann wiederum mehrere subrecords enthalten(siehe FORMAT06.pdf Seite: 6-8), die auch mehrfach in einer Log-Datei vorkommen k�nnen. Ein Beispiel eines h�ufig vorkommenden subrecords von @BTEST w�re @BLOCK.
\subsection{Besonderheiten und Bemerkungen}\label{Besonderheiten}
\begin{itemize}
\item In der Dokumentation des Agilent Data Formats (FORMAT06.pdf) ist @BS-CON als Kind des @BTEST-Records als auch des @BLOCK-Records  aufgef�hrt und taucht in den und zur Verf�gung stehenden Log-Dateien an beiden in Frage kommenden Stellen auf. Um das Problem zu l�sen und @BS-CON einem Knoten eindeutig zuordnen zu k�nnen, m�sste man in jeder Log-Datei herausfinden, ob @BS-CON in der Hierarchie auf derselben H�he steht wie @BLOCK oder ein Kind von @BLOCK ist. 
\item Des Weiteren stimmt die Anzahl der Attribute f�r das @BATCH-Record und f�r das @BTEST-Record in der Dokumentation nicht mit der Legende und auch nicht mit den tats�chlichen Log-Daten �berein.  In der Dokumentation hat @BATCH 13 Attribute, in der Legende und in den Log-Daten jeweils 14 Attribute. Das @BTEST-Record hat 13 Attribute in der Dokumentation und in der Legende, aber in den Log-Daten sind es nur 12.
\item Es existieren mehrere Gruppen von Pr�fixen. Log-records, die beispielsweise mit ''@A'' beginnen, wurden mit analogen Test-statements, und die, die mit ''@D'' beginnen, wurden mit digitalen Test-statements erzeugt. 
\item Jedes @BTEST sowie jedes @BLOCK-Record und alle Records, die auf der selben Hierarchiestufe angesiedelt sind beginnen in einer neuen Zeile der Log-Datei.
\item Die Datenfelder zwischen den Pipes k�nnen von folgenden Typ sein: 
\begin{itemize}
\item Boolean (1 oder 0)
\item Flie�kommazahl(z.B. 1.0253+E01)
\item Integer(z.B. -125)
\item String(z.B. Node14, +5Volts o.�.) 
\item Null-Zeichen (Ein leeres Datenfeld enth�lt kein einziges Zeichen zwischen zwei Pipes. Dabei ist zu beachten, dass hinten dem letzten Pipe sich immer noch ein Datenfeld befindet, das keine Werte enth�lt.) 
\end{itemize}
\end{itemize}


