\section{Das Agilent Format}\label{AgilentFormat}
\subsection{Struktur und Syntax}
Testdaten im Agilent-Format  werden in einer Datei als eine Folge von s.g. log-records gespeichert. Jedes log-record ist durch geschweifte Klammern umschlossen und beginnt mit einen Pr�fix, welches aus einem @-Zeichen besteht, gefolgt von beschreibenden Zeichen, die den Typ des Records eindeutig identifizieren.  Darauf folgt eine bestimmte Anzahl von Datenfeldern, die alphanumerische Messwerte im ASCII-Format enthalten und voneinander durch das \textbar-Zeichen(Pipe) getrennt sind.  Sollten einem Datenfeld keine Messwerte zugeordnet worden sein, wird dies vom Testsystem durch ein Leerzeichen zwischen den Pipes dargestellt.
Die log-records sind hierarchisch angeordnet in records und subrecords. Die subrecords dienen dazu  das vorhergehende record genauer zu beschreiben.  Dabei ist jedem subrecord h�chstens ein record unmittelbar �bergeordnet und somit kann man die Log-Datei als einen Baum darstellen um zwischen den einzelnen Knoten zu navigieren und sie auszulesen. \\
Das Wurzelelement jeder der uns zur Verf�gung stehenden Log-Datei ist das @BATCH-Record, welches erst vollst�ndig durch das @BTEST-Record beschrieben wird und dieses als subrecord mit geschweiften Klammern umschlie�t. Jede Log-Datei enth�lt jeweils ein @BATCH-Record und ein @BTEST-Record.  Dies ist die �bliche Struktur  und sie wurde in allen der uns zur Verf�gung stehenden Log-Dateien befolgt. Je nach Testsystem k�nnte es jedoch evtl. zu Abweichungen von dieser Struktur kommen. Es k�nnten beispielsweise mehrere @BATCH-Records in einer Log-Datei gespeichert werden.   
Das @BTEST-Record kann wiederum mehrere subrecords enthalten(Siehe Agilent Data Format), die auch mehrfach in einer Log-Datei vorkommen k�nnen. Ein Beispiel eines h�ufig vorkommenden subrecords von @BTEST w�re @BLOCK.
\subsection{Besonderheiten und Bemerkungen}
In der Dokumentation des Agilent Data Formats ist @BS-CON als Kind des @BTEST-Records als auch des @BLOCK-Records  aufgef�hrt und taucht in den Log-Dateien an beiden in Frage kommenden Stellen auf, was den Aufwand bei der Implementierung des Parsers deutlich erh�ht. Man muss herausfinden ob @BS-CON in der Hierarchie auf derselben H�he steht wie @BLOCK oder ein Kind von @BLOCK ist.
\newline
Des Weiteren stimmt die Anzahl der Attribute f�r das @BATCH-Record und f�r das @BTEST-Record in der Dokumentation nicht mit der Legende und auch nicht mit den tats�chlichen Log-Daten �berein.  In der Dokumentation hat @BATCH 13 Attribute, in der Legende und in den Log-Daten jeweils 14 Attribute. Das @BTEST-Record hat 13 Attribute in der Dokumentation und in der Legende, aber in den Log-Daten sind es nur 12. 


