\section{Unit Tests}\label{UnitTests}
\subsection{ZeilenParserTest}
Im ZeilenparserTest werden die Methode \textit{parseZeile()}, \textit{hasMoreStringNodes()} sowie \textit{getNextNode()} getestet. Zuerst wird der Methode parseZeile() eine Zeile der Log-Datei als String �bergeben z.B.:
\begin{verbatim}
{@BATCH|501-6338-02|50|12|1||btest|040107103921||solmb3t1|||||
\end{verbatim}
Daraufhin werden alle geschweiften Klammern aufgel�st und die einzelnen Parameter in einer List$<$String$>$ gespeichert. Die Methode hasMoreStringNodes() liefert true, falls ein Listenelement den Anfang eines Records darstellt und mit @-Zeichen beginnt. Wenn eine Zeile mehrere Records enth�lt, wird mit getNextNode() immer das komplette n�chste Record als Liste zur�ckgegeben. Diese Funktionalit�t wird beispielsweise in folgendem Test gepr�ft:
\begin{verbatim}
@Test
	public void testGetNextSecond() {
		String inpu = "{@A-JUM|0|+7.630803E+06{@LIM2|+9.999999E+99|+1.000000E+04}}";
		zeilenParser.parseZeile(inpu);
		zeilenParser.getNextNode();
		ArrayList<String> stringNode = new ArrayList<String>();
		stringNode.add("@LIM2");
		stringNode.add("+9.999999E+99");
		stringNode.add("+1.000000E+04");
		assertTrue(zeilenParser.getNextNode().equals(stringNode));
	} 
\end{verbatim}
\subsection{KnotenErzeugerTest}
Im KnotenErzeugerTest wird die Methode \textit{erzeugeKnoten()} getestet. Ihr wird eine Liste mit den Attributen eines Records �bergeben, die mit Hilfe der \textit{NodeCreatorUtil}-Klasse erzeugt wurde. Daraufhin pr�ft die Methode anhand des ersten Listenelements um welches Record es sich handelt und erzeugt eine Instanz der entsprechenden Record-Klasse, die eine Implementierung der \textit{ILogEntry}-Schnittstelle ist. Jede Record-Instanz enth�lt die beiden Listen \textbf{headings} und \textbf{values}. Headings sind die Namen der Record-Attribute und werden im Konstruktor gesetzt. Anschlie�end ruft die \textit{erzeugeKnoten()}-Methode createLogEntryNode() auf, in der die values gesetzt werden und eine Instanz der \textit{LogEntryNode-Klasse} erzeugt wird. Im Unit-Test wird gepr�ft, ob der richtige Knoten erfolgreich erzeugt wurde:
\begin{verbatim}
@Test
	public void testErzeugeKnotenBATCH() {
		ILogEntryNode logEntryNode = testNodeCreator.createNodeBATCH();
		assertEquals(BATCH.class, logEntryNode.getLogEntry().getClass());
	}
\end{verbatim}

\subsection{KnotenEinhaengenTest}
Hierbei wird die Methode \textit{addKnoten()} getestet. Nachdem ein Knoten erzeugt wurde, wird er ihr als Parameter �bergeben. Diese pr�ft zuerst um welchen Knoten-Typ es sich handelt. Wenn es ein BATCH-Knoten ist, wird er zum Wurzelknoten(root) des Baumes. Der BTEST-Knoten wird mit \textit{root.getSubNodes().add(knoten)} an den Wurzelknoten eingeh�ngt. Folgt anschlie�end ein BLOCK-Knoten, werden zuerst an alle seine Unterknoten deren Kinder eingeh�ngt und zum Schluss wird der BLOCK-Knoten in \textit{flushBlockTemp()} an den BTEST-Knoten angef�gt. 
Im KnotenEinhaengenTest werden zuerst der Reihe nach die gew�hlten Knoten an das Wurzelelement mit addKnoten() eingeh�ngt um anschlie�end ein bestimmtes Element herauszulesen:
\begin{verbatim}
@Test
	public void testAddKnotenBATCH_BTEST_BLOCK_AJUM() {
		knotenEinhaenger.addKnoten(testNodeCreator.createNodeBATCH());
		knotenEinhaenger.addKnoten(testNodeCreator.createNodeBTEST());
		knotenEinhaenger.addKnoten(testNodeCreator.createNodeBLOCK());
		knotenEinhaenger.addKnoten(testNodeCreator.createNodeAJUM());
		knotenEinhaenger.flush();
		ILogEntryNode node = knotenEinhaenger.getRoot();
		ILogEntryNode batch = node;
		ILogEntryNode btest = batch.getSubNodes().get(0);
		ILogEntryNode block0 = btest.getSubNodes().get(0);
		ILogEntryNode block0_ajum = block0.getSubNodes().get(0);
		assertEquals(AJUM.class, block0_ajum.getLogEntry().getClass());
	}
\end{verbatim}