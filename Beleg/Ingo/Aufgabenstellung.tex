\section{Aufgabenstellung}\label{Aufgabenstellung}
In diesem Projekt sollen gegebene Agilent-Logdateien geparst werden. Aus den geparsten Zeilen soll ein Baum in einem Intermediate-Format erstellt werden. Der erzeugte Baum wird im Hauptspeicher des Rechners gehalten. Die Weiterverarbeitung des Baumes im Intermediate-Format erfolgt in einem anderen Projekt und ist nicht Gegenstand dieser Arbeit. In diesem Projekt muss das Intermediate-Format nicht entworfen werden. Statt dessen wird es von Felix Deutschmann und Daniel Horbach �bernommen. Ein Baum im Intermediate-Format wird immer nur aus einer Agilent-Logdatei erzeugt, d.h. es soll nicht ein Baum aus zwei oder mehreren Agilent-Logdateien erzeugt werden. $Agilent-Logdatei -> Agilent-Parser -> Baum im Intermediate-Format$. 

Das Frontend soll nur Knotennamen ber�cksichtigen, die in den zur Verf�gung stehenden Agilent-Logdateien auch vorkommen. Andere Knotennamen brauchen im Frontend nicht implementiert werden. Treten bei der Verarbeitung einer Agilent-Logdatei einmal unerwartete Knotennamen auf, werden diese in der Datei UnsupportedNodeNames.txt weggeschrieben.

Die Reihenfolge der Kindknoten spielt bei der Erstellung des Baumes keine Rolle.