\section{Nichtfunktionale Anforderungen}\label{NichtfunktionaleAnforderungen}
Die Anwendung wird test-driven entwickelt. F�r jede nach au�en sichtbare Funktionalit�t einer Klasse werden Unit-Tests geschrieben. Es werden Funktionstests geschrieben, die den Parser funktions�bergreifend testen. Daneben dokumentieren funktions�bergreifende Tests, wie der Parser verwendet werden sollte - d.h. welche Methoden aufgerufen werden k�nnen und was diese Methoden zur�ckgeben. Die Anwendung soll au�erdem wartungsfreundlich und damit leicht �nderbar und anpassbar sein, wenn z.B. neue Knotennamen hinzugef�gt werden. Es wird auf sprechende Variablennamen und Methodennamen geachtet. Es wird darauf geachtet, dass jede Methode nur eine Funktionalit�t implementiert und somit kurz und �bersichtlich bleibt. Es wird Dependency Injection eingesetzt, um die Objektinstanziierung von der Programmlogik zu trennen. Es wird Maven eingesetzt, um die Anwendung automatisiert erstellen und testen zu k�nnen.