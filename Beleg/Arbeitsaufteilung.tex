\newpage
\chapter{Arbeitsaufteilung}


\begin{table}[h] \begin{flushleft}  \begin{tabular}{|l||c|c|c|c|c|c|}
\hline
\textbf{Arbeit}		&	\textbf{C. Ochmann}	& \textbf{I. K�rner}  \\ \hline \hline
Abstract   	      &                     & 0       \\
Einleitung  &                             		      & ~\ref{Einleitung} \\
Aufgabenstellung&                                  & ~\ref{Aufgabenstellung}  \\
Forschungsgegenstand&                              & ~\ref{RelevanzDesForschungsgegenstandes} \\ 
akt. Wissensstand&                                      & ~\ref{DerAktuelleWissensstand}  \\ 
Hintergrund&                                      & ~\ref{Hintergrund}  \\ 
Analyse&                                      & ~\ref{Analyse}  \\ 
Funktionale Anforderungen&                                      & ~\ref{FunktionaleAnforderungen}  \\ 
Nichtfunktionale Anforderungen&                                      & ~\ref{NichtfunktionaleAnforderungen}  \\ 
Entwurf&                                      & ~\ref{Entwurf}  \\ 
Struktur und Syntax& ~\ref{strukturundsyntax} &										\\
Besonderheiten und Bemerkungen&  ~\ref{Besonderheiten}    &     \\
Entwicklungsumgebung&                                      & ~\ref{Entwicklungsumgebung}  \\ 
Dependency Injection mit Goolge Guice&                                      & ~\ref{DependencyInjectionMitGoogleGuice}  \\ 
Projekt importieren&                                      & ~\ref{ProjektImportieren}  \\ 
Projekt ausf�hren&                                      & ~\ref{ProjektAusfuehren}  \\ 
ZeilenParserTest&   ~\ref{zeilenparsertest}          &  \\
KnotenErzeugerTest&   ~\ref{knotenerzeugertest}          &  \\
KnotenEinhaengenTest&   ~\ref{knoteneinhaengentest}          &  \\
Funktionale Tests&                                      & ~\ref{FunktionaleTests}  \\ 
Zusammenfassung&                                      & ~\ref{Zusammenfassung}  \\ 
\hline \hline
\end{tabular} \end{flushleft} \caption{Aufteilung} \end{table}

